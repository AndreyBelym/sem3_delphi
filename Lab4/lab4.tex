\input template.tex
\begin{document}
\selectlanguage{russian}
\maketitle 4 {ПРОГРАММИРОВАНИЕ С ИСПОЛЬЗОВАНИЕМ СТРОК}
\setcounter{page}{2}
\normalfont
\ssec{Цель работы}
Цель работы заключается в том, чтобы изучить правила работы с компонентами
TListBox и TСomboBox и написать программу, работающую со строками.

\ 
\ssec{Задание на работу}
Дана строка, состоящая из групп нулей и единиц. Найти и вывести на экран
группы с четным количеством символов.

\clearpage
\ssec{Теоретическая справка}
{\hbadness=10000
\loop{}\ifnum\thepage<5{\hrule\hfill\\}\repeat
\newcounter{lc}\setcounter{lc}{29}
\loop{}\ifnum\value{lc}>0{\addtocounter{lc}{-1}\hrule\hfill\\}\repeat
}
\ssec{Схема алгоритма}
На рисунке \ref{LAB1} представлена схема общего алгоритма ввода данных,  получения из строки и вывода групп из нулей и единиц 
с четным количеством символов.
\pic{LAB1.png}{Схема общего алгоритма  ввода данных, получения и вывода групп из нулей и единиц}{LAB1}{H}

На рисунке \ref{LAB2} представлена схема алгоритма получения из заданой строки групп из нулей и единиц 
с четным количеством символов.
\pic{LAB2.png}{Схема алгоритма  получения групп из нулей и единиц}{LAB2}{H}

На рисунке \ref{LAB3} представлена схема алгоритма добавления возможной группы из нулей и единиц с четным количеством символов в динамический массив.
\pic{LAB3.png}{Схема алгоритма добавления группы из нулей и единиц массив}{LAB3}{H}

На рисунке \ref{LAB4} представлена схема алгоритма ввода строки для поиска в ней групп из нулей и единиц 
с четным количеством символов.
\pic{LAB4.png}{Схема алгоритма  ввода строки для поиска в ней групп из нулей и единиц}{LAB4}{H}

На рисунке \ref{LAB5} представлена схема алгоритма вывода групп из нулей и единиц 
с четным количеством символов на экран.
\pic{LAB5.png}{Схема алгоритма  вывода групп из нулей и единиц}{LAB5}{H}
%\vfill
\clearpage
\ssec{Инструкция пользователю}
Программа позволяет найти в заданной строке групп из нулей и единиц с четным количеством символов.

Для того, чтобы передать программе строку, введите её в соответствующее поле и нажмите клавишу Enter. Далее щелкните по нужной строке в списке ранее введенных  строк, и тогда в правом поле вывода программа отобразит в столбик найденные  группы из нулей и единиц с четным количеством символов. Ранее введенные строки можно удалять из списка путем нажатия соотвествующей клавиши. Для завершения работы программы нажмите кнопку "Выход".
\ 
\ssec{Инструкция программиста}
При создании программы  получения из строки групп из нулей и единиц 
с четным количеством символов сначала был ообъявлен тип TStrArray как "array of string", описывающий динамический массив строк, в которых будут хранится найденные группы.

Далее программа была разбита на следующие подпрограммы:

\elist{
\item Функция getGroups - получение из строки групп из нулей и единиц 
с четным количеством символов.
Возвращает динамический массив групп.

function getGroups(const str:String):TStrArray;

Параметры  функции \ftab{getGroups:1}:
\tabl{Параметры  функции получения групп из нулей и единиц}{
\tabln{str&string & строка для поиска групп}
}{getGroups:1}{H}

Локальные переменные  функции \ftab{getGroups:2}:
\tabl{Локальные переменные  функции  получения групп из нулей и единиц}{
\tabln{x&TStrArray &динамический массив групп}
\tabln{beg&integer & начало группы символов}
\tabln{i&integer & индекс текущего элемента строки}
}{getGroups:2}{H}

\item Процедура addGroup - добавление возможной группы из нулей и единиц с четным количеством символов в динамический массив.

procedure addGroup(const str:string;const beg,end\_:integer;var x:TStrArray);

Параметры процедуры \ftab{addGroup:1}:
\tabl{Параметры процедуры добавления возможной группы из нулей и единиц}{
\tabln{str&String & строка с группой символов}
\tabln{beg,end\_&integer & начальный и конечный индексы группы}
\tabln{x&TStrArray & динамический массив групп}
}{addGroup:1}{H}


Локальные переменные  процедуры \ftab{addGroup:2}:
\tabl{Локальные переменные  процедуры добавления возможной группы из нулей и единиц}{
\tabln{k&integer & размер массива групп}
}{addGroup:2}{H}

\item Функция InputString - получение строки для поиска в ней групп из нулей и единиц 
с четным количеством символов.

function TfrmMain.InputString:string;

\item Процедура Output - вывод групп из нулей и единиц 
с четным количеством символов на форму.

procedure TfrmMain.Output(x:TStrArray);

Параметры процедуры \ftab{Output:1}:
\tabl{Параметры процедуры вывода групп из нулей и единиц на форму}{
\tabln{x&TStrArray & динамический массив групп}
}{Output:1}{H}


Локальные переменные  процедуры \ftab{Output:2}:
\tabl{Локальные переменные  процедуры  вывода групп из нулей и единиц на форму}{
\tabln{i&integer & переменная-счетчик}
}{Output:2}{H}

\item Обработчик события щелчка мышкой на списке lstStringsClick - основная процедура программы.

procedure TfrmMain.lstStringsClick(Sender: TObject);

Параметры процедуры \ftab{lstStringsClick:1}:
\tabl{Параметры процедуры-обработчика события щелчка мышкой на списке}{
\tabln{Sender&TObject & объект-возбудитель события}
}{lstStringsClick:1}{H}


Локальные переменные  процедуры \ftab{lstStringsClick:2}:
\tabl{Локальные переменные  процедуры-обработчика события щелчка мышкой на списке}{
\tabln{x&TStrArray & динамический массив групп}
\tabln{str&string & строка для поиска групп}
}{lstStringsClick:2}{H}

\item Обработчик события нажатия кнопки удаления строки btnDeleteClick.

procedure TfrmMain.btnDeleteClick(Sender: TObject);

Параметры процедуры \ftab{btnDeleteClick:1}:
\tabl{Параметры процедуры-обработчика события нажатия кнопки удаления строки}{
\tabln{Sender&TObject & объект-возбудитель события}
}{btnDeleteClick:1}{H}

\item Процедура edtNewLineKeyPress - обработчик нажатия клавиши в поле ввода 
новой строки edtNewLine.

procedure TfrmMain.edtNewLineKeyPress(Sender: TObject; var Key: Char);

Параметры процедуры \ftab{edtNewLineKeyPress:1}:
\tabl{Параметры  процедуры-обработчика события нажатия кнопки удаления строки}{
\tabln{Sender&TObject & объект-возбудитель события}
}{edtNewLineKeyPress:1}{H}
}
\clearpage
\ssec{Текст программы}
Ниже представлен текст программы на языке Delphi 7, реализующей получение из строки групп из нулей и единиц 
с четным количеством символов.
\prog{Delphi}{UnitMain.pas}
\clearpage

\ 
\ssec{Тестовый пример}
Ниже на рисунке \ref{SCR1} представлен пример работы программы для строки, не имеющей в себе групп из нулей и единиц с четным количеством символов.
\pic{SCR1.png}{Пример работы программы  строки, не имеющей нужных  групп из нулей и единиц}{SCR1}{H}
Ниже на рисунке \ref{SCR2} представлен пример работы программы для строки, в которой содержатся группы из нулей и единиц с четным количеством символов.
\pic{SCR2.png}{Пример работы программы  для строки, в которой содержатся нужные группы из нулей и единиц}{SCR2}{H}
\ 
\ssec{Вывод}
В этой лабораторной работе я изучил компоненты TListBox и TComboBox, а также научился раотать со строками в среде Delphi. Delphi предоставляет несколько способов представления строковых переменных и констант, эффективных для обработки с точки зрения машины и удобных для работы с точки зрения программиста. Также язык предлагает всевозможные встроенные функции для многочисленных преобразований строковой информации. А визуальные компоненты TListBox и TComboBox позволяют удобным образом пользователю взаимодействовать с программой при вводе, выводе и анализе текстовой информации.
\end{document}
