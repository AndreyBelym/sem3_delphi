\input template.tex
\begin{document}
\selectlanguage{russian}
\maketitle 5 {ПРОГРАММИРОВАНИЕ С ИСПОЛЬЗОВАНИЕМ ЗАПИСЕЙ И ФАЙЛОВ}
\setcounter{page}{2}
\normalfont
\ssec{Цель работы}
Цель работы заключается в том, чтобы изучить правила работы с компонентами TOpenDialog и TSaveDialog и написать программу с использованием файлов и данных типа "запись". 

\ 
\ssec{Задание на работу}
В справочной автовокзала хранится расписание движения автобусов. Для каждого рейса указаны его номер, тип автобуса, пункт назначения, время отправления и прибытия. Вывести информацию о рейсах, которыми можно воспользоваться для прибытия в пункт назначения раньше заданного времени.

\clearpage
\ssec{Теоретическая справка}
{\hbadness=10000
\loop{}\ifnum\thepage<5{\hrule\hfill\\}\repeat
\newcounter{lc}\setcounter{lc}{29}
\loop{}\ifnum\value{lc}>0{\addtocounter{lc}{-1}\hrule\hfill\\}\repeat
}
\ssec{Схема алгоритма}
На рисунке \ref{LAB1} представлена схема общего алгоритма вывода информации о рейсах, которыми можно воспользоваться для прибытия в пункт назначения раньше заданного времени.
\pic{LAB1.png}{Схема общего алгоритма вывода информации о рейсах}{LAB1}{H}
На рисунке \ref{LAB2} представлена схема алгоритма загрузки расписания из файла и инициализации индексного массива.
\pic{LAB2.png}{Схема алгоритма загрузки расписания из файла}{LAB2}{H}
На рисунке \ref{LAB3} представлена схема алгоритма поиска рейсов, которыми можно воспользоваться для прибытия в пункт назначения раньше заданного времени,.
\pic{LAB3.png}{Схема алгоритма поиска рейсов}{LAB3}{H}
На рисунке \ref{LAB4} представлена схема алгоритма сохранения найденных маршрутов в файл.
\pic{LAB3.png}{Схема алгоритма сохранения маршрутов}{LAB4}{H}
%\vfill
\clearpage
\ssec{Инструкция пользователю}
Программа позволяет создавать, редактировать, открывать и сохранять расписания автобусных маршрутов, а также осуществлять вывод информации о рейсах, которыми можно воспользоваться для прибытия в пункт назначения раньше заданного времени.

Сверху в окне программы располагаются кнопки создания нового расписания, открытия и сохранения расписания. 

Основную часть окна занимает метка расписания. Самая верхняя строка сетки подсказывает предназначение каждой ячейки таблицы. Любую ячейку можно свободно редактировать. При редактировании ячеек с датами появляется вспомогательное окошко, в котором можно выбрать дату и время; если вы хотите ввести данные параметры вручную, нажмите кнопку "Отмена". При неправильном вводе при попытке любых операций с таблицей программа сообщит координаты ячеек с неправильными значениями.


Нижнюю часть программы занимают элементы фильтрации, которые позволяют осуществить  вывод информации о рейсах, которыми можно воспользоваться для прибытия в пункт назначения раньше заданного времени. Просто введите нужные параметры и нажмите кнопку "Отфильтровать", чтобы просмотреть результаты; кнопку "Сбросить фильтр", чтобы получить полное расписание и кнопку "Сохранить" для сохранения найденных маршрутов в файл.

В случае возникновения непредвиденной ошибки программа выводит сообщение и блокирует все элементы управления, кроме кнопок создания и открытия расписания. Если не получается открыть или создать расписание, перезапустите программу. 
\ 
\ssec{Инструкция программиста}
При создании программы  вывода информации о рейсах, которыми можно воспользоваться для прибытия в пункт назначения раньше заданного времени, были объявлены типы:
\elist{
\item TRoute - запись, которая описывает маршрут, и  состоит из полей, представленных в таблице \ref{TRoute:1}.

\tabl{Поля типа "маршрут" }{
\tabln{Number&Word&Номер маршрута}
\tabln{BusType&String[25]&Тип автобуса}
\tabln{ Destination&String[50]&Пункт назначения}
\tabln{OutTime,InTime&TDateTime&Время выезда и прибытия}
}{TRoute:1}{H}
 \item TRouteTable, обьявлен как "array of TRoute"  и описывает расписание маршрутов;
 \item TIndexArray, обьявлен как "array of Integer"  и описывает индексный массив расписания.
}

Также описан класс THackStringGrid=class(TStringGrid), который описывает вариант таблицы TStringGrid с некоторыми дополнительными полезными методами.

Были объявлены следующие глобальные переменные:

\tabl{Глобальные переменные модуля}{
\tabln{ timetable&TRouteTable&Расписание маршрутов}
\tabln{ index&TIndexArray&Индексный массив расписания}
}{GlobalVar:1}{H}

Далее программа была разбита на следующие подпрограммы:

\elist{
\item Процедура ReadFromFile считывает из файла с именем filename расписание timetable
и создает индексный массив index.
Может возбуждать исключения в случае возникновения ошибок.

procedure ReadFromFile(filename:string;var timetable:TRouteTable;

 var index:TIndexArray);

Параметры процедуры \ftab{ReadFromFile:1}:
\tabl{Параметры процедуры чтения расписания}{
\tabln{filename&string&имя файла для чтения}
\tabln{var timetable&TRouteTable&расписание маршрутов}
\tabln{var index&TIndexArray&индексный массив расписания}
}{ReadFromFile:1}{H}


Локальные переменные  процедуры \ftab{ReadFromFile:2}:
\tabl{Локальные переменные  процедуры чтения расписания}{
\tabln{f&file of TRoute&файл для чтения}
\tabln{i&Integer&счетчик для обработки массивов}
}{ReadFromFile:2}{H}

\item Функция Filter отфильтровывает в расписании table маршруты, приходящие
в пункт dest раньше даты date. Возвращает индексы таких маршрутов.
Может возбуждать исключения в случае возникновения ошибок.

function Filter(table:TRouteTable;date:TDateTime;dest:String):TindexArray;

Параметры  функции \ftab{Filter:1}:
\tabl{Параметры  функции фильтрации расписания}{
\tabln{table&TRouteTable&расписание маршрутов}
\tabln{date&TDateTime&максимальная дата прибытия}
\tabln{dest&String&имя пункта назначения}

}{Filter:1}{H}

Локальные переменные  функции \ftab{Filter:2}:
\tabl{Локальные переменные  функции фильтрации расписания}{
\tabln{x&TIndexArray&найденные маршруты}
\tabln{i,k&Integer&счетчики для обработки массивов}
}{Filter:2}{H}

\item Процедура SaveToFile сохраняет в файл с именем filename строки из расписания
timetable с номерами из массива index.
Может возбуждать исключения в случае возникновения ошибок.

procedure SaveToFile(filename:string; timetable:tRouteTable; index:TIndexArray);

Параметры процедуры \ftab{SaveToFile:1}:
\tabl{Параметры процедуры сохранения расписания}{
\tabln{filename&string&имя файла для записи}
\tabln{timetable&tRouteTable&расписание маршрутов}
\tabln{index&TIndexArray&индесы строк расписания для записи}
}{SaveToFile:1}{H}


Локальные переменные  процедуры \ftab{SaveToFile:2}:
\tabl{Локальные переменные  процедуры сохранения расписания}{
\tabln{f&file of TRoute&файл для записи}
\tabln{i&Integer&счетчик для обработки массивов}
}{SaveToFile:2}{H}






\item Процедура btnDelLineClick. Обработчик нажатия кнопки удаления строки.
   
 procedure TfrmMain.btnDelLineClick(Sender: TObject);

Параметры процедуры \ftab{btnDelLineClick:1}:
\tabl{Параметры процедуры-обработчика события нажатия кнопки удаления строки}{
\tabln{Sender&TObject & объект-возбудитель события}
}{btnDelLineClick:1}{H}

\item Процедура btnAddLineClick. Обработчик нажатия кнопки добавления строки.

 procedure  TfrmMain.btnAddLineClick(Sender: TObject);

Параметры процедуры \ftab{btnAddLineClick:1}:
\tabl{Параметры процедуры-обработчика события нажатия кнопки добавления строки}{
\tabln{Sender&TObject & объект-возбудитель события}
}{btnAddLineClick:1}{H}

\item Процедура FormCreate. Обработчик события создания формы.

 procedure  TfrmMain.FormCreate(Sender: TObject);

Параметры процедуры \ftab{FormCreate:1}:
\tabl{Параметры процедуры-обработчика события  создания формы}{
\tabln{Sender&TObject & объект-возбудитель события}
}{FormCreate:1}{H}

  \item Процедура btnFilterClick.  Обработчик нажатия кнопки фильтрации.
   
 procedure  TfrmMain.btnFilterClick(Sender: TObject);

Параметры процедуры \ftab{btnFilterClick:1}:
\tabl{Параметры процедуры-обработчика события нажатия  кнопки фильтрации}{
\tabln{Sender&TObject & объект-возбудитель события}
}{btnFilterClick:1}{H}
   
\item Процедура UpdateGrid. Метод обновления таблицы из расписания.
    
procedure  TfrmMain.UpdateGrid;
    
\item Процедура updatetable. Метод обновления расписания из таблицы.
    
procedure  TfrmMain.updatetable;
    
\item Процедура btnNoFilterClick. Обработчик нажатия кнопки отмены фильтрации.
    
procedure  TfrmMain.btnNoFilterClick(Sender: TObject);

Параметры процедуры \ftab{btnNoFilterClick:1}:
\tabl{Параметры процедуры-обработчика события нажатия кнопки отмены фильтрации}{
\tabln{Sender&TObject & объект-возбудитель события}
}{btnNoFilterClick:1}{H}
    
\item Процедура tbtNewClick. Обработчик нажатия кнопки создания расписания.
    
procedure TfrmMain.tbtNewClick(Sender: TObject);

Параметры процедуры \ftab{tbtNewClick:1}:
\tabl{Параметры процедуры-обработчика события нажатия кнопки создания расписания}{
\tabln{Sender&TObject & объект-возбудитель события}
}{tbtNewClick:1}{H}
    
\item Процедура tbtSaveClick. Обработчик нажатия кнопки сохранения расписания.
    
procedure  TfrmMain.tbtSaveClick(Sender: TObject);

Параметры процедуры \ftab{tbtSaveClick:1}:
\tabl{Параметры процедуры-обработчика события нажатия кнопки  сохранения расписания}{
\tabln{Sender&TObject & объект-возбудитель события}
}{tbtSaveClick:1}{H}
    
\item Процедура tbtOpenClick. Обработчик нажатия кнопки открытия расписания.
    
procedure  TfrmMain.tbtOpenClick(Sender: TObject);

Параметры процедуры \ftab{tbtOpenClick:1}:
\tabl{Параметры процедуры-обработчика события нажатия кнопки открытия расписания}{
\tabln{Sender&TObject & объект-возбудитель события}
}{tbtOpenClick:1}{H}
    
\item Процедура sgdTimetableMouseUp. Обработчик отпускания клавиши мыши на таблице.
    
procedure TfrmMain.sgdTimetableMouseUp ( Sender: TObject; 
	
Button: TMouseButton;
      Shift: TShiftState; X, Y: Integer);

Параметры процедуры \ftab{sgdTimetableMouseUp:1}:
\tabl{Параметры процедуры-обработчика события отпускания клавиши мыши на таблице}{
\tabln{Sender&TObject & объект-возбудитель события}
\tabln{Button&TMouseButton &состояние кнопок мыши}
\tabln{Shift&TShiftState & состояние кнопки Shift}
\tabln{X, Y&Integer &координаты курсора}
}{sgdTimetableMouseUp:1}{H}
   
\item btnSaveFilteredClick. Обработчик нажатия кнопки сохранения отфильтрованных маршрутов.

    procedure  TfrmMain.btnSaveFilteredClick(Sender: TObject);

Параметры процедуры \ftab{btnSaveFilteredClick:1}:
\tabl{Параметры процедуры-обработчика события нажатия кнопки сохранения отфильтрованных маршрутов}{
\tabln{Sender&TObject & объект-возбудитель события}
}{btnSaveFilteredClick:1}{H}
    
\item Процедура FormDestroy. Обработчик события уничтожения формы.

    procedure  TfrmMain.FormDestroy(Sender: TObject);

Параметры процедуры \ftab{FormDestroy:1}:
\tabl{Параметры процедуры-обработчика событияуничтожения формы}{
\tabln{Sender&TObject & объект-возбудитель события}
}{FormDestroy:1}{H}
}

\clearpage
\ssec{Текст программы}
Далее приведен текст программы  вывода информации о рейсах, которыми можно воспользоваться для прибытия в пункт назначения раньше заданного времени, написанной на языке Delphi 7.
\prog{Delphi}{UnitMain.pas}
\clearpage

\ 
\ssec{Тестовый пример}
Ниже на рисунке \ref{SCR1} представлен пример работы программы при составлении расписания маршрутов.
\pic{SCR1.png}{Пример работы программы при составлении расписания}{SCR1}{H}
Ниже на рисунке \ref{SCR2} представлен пример работы программы, отображающей маршруты, прибывающие до 13:39 в Москву для указанного выше расписания.
\pic{SCR2.png}{Пример результатов поиска маршрутов программой}{SCR2}{H}
\ 
\ssec{Вывод}
В ходе выполнения данной лабораторной работы я научился использовать 
файлы  в программах на Delphi. Delphi расширяет стандартный набор функций и процедур для работы с файлами, заимствованный из Паскаля, и кроме того, предоставляет удобные визуальные интерфейсы для организации помощи пользователю при работе с фалами.  Файлы   позволяют 
долговременно   хранить   большие   объёмы   данных   и   передавать   их   между 
совершенно различными программами. К сожалению, работа с файлами ведется 
медленнее, чем с оперативной памятью.

\end{document}
