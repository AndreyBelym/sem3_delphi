\input template.tex
\begin{document}
\selectlanguage{russian}
\maketitle 2 {ПРОГРАММИРОВАНИЕ С ИСПОЛЬЗОВАНИЕМ ПЕРЕКЛЮЧАТЕЛЕЙ. ОТЛАДКА ПРОГРАММЫ.}
\setcounter{page}{2}
\normalfont
\ssec{Цель работы}
Цель работы заключается в том, чтобы научиться  пользоваться  простейшими 
компонентами  организации  переключений  (TСheckBox,  TRadioGroup), а также  написать  и 
отладить программу разветвляющегося алгоритма.  


\ 
\ssec{Задание на работу}
Рассчитать значение функции:
$$ w=|\cos x - \cos y|^{(1+2\sin^2 y)}\left[1+z+{{z^2}\over 2}+{{z^3}\over 3}+{{z^4}\over 4}\right] $$

 В качестве f(x) использовать по выбору: $sinh(x)$, $x^2$, $e^x$. 

\clearpage
\ssec{Теоретическая справка}
{\hbadness=10000
\loop{}\ifnum\thepage<5{\hrule\hfill\\}\repeat
\newcounter{lc}\setcounter{lc}{29}
\loop{}\ifnum\value{lc}>0{\addtocounter{lc}{-1}\hrule\hfill\\}\repeat
}
\ssec{Схема алгоритма}
На рисунке \ref{alg:a} представлена схема общего алгоритма расчета значения сложной функции
\pic{LAB1.png}{Схема общего алгоритма   расчета значения сложной функции}{alg:a}{H}
На рисунке \ref{alg:b} представлена схема алгоритма ввода данных для  расчета значения сложной функции.
\pic{LAB2.png}{Схема алгоритма ввода данных}{alg:b}{H}
На рисунке \ref{alg:c} представлена схема алгоритма расчета значения сложной функции.
\pic{LAB3.png}{Схема алгоритма   расчета значения сложной функции}{alg:c}{H}
На рисунке \ref{alg:d} представлена схема алгоритма вывода рассчитанного значения сложной функции.
\pic{LAB4.png}{Схема алгоритма вывода результата}{alg:d}{H}

%\vfill
\clearpage
\ssec{Инструкция пользователю}
Программа позволяет рассчитать значение функции:
$$ w=|\cos x - \cos y|^{(1+2\sin^2 y)}\left[1+z+{{z^2}\over 2}+{{z^3}\over 3}+{{z^4}\over 4}\right] $$
,
где f(x) по выбору может быть: $sinh(x)$, $x^2$, $e^x$.

Дл\ расчета значения функции сначала выберите вспомогательную функцию, установив переключатель в позицию напротив выбранной функции. Далее введите числа x и y, от которых рассчитывается значение. После ввода данных нажмите кнопку ``Посчитать''. Если в данных были допущены ошибки, программа сообщит об этом во всплывающем окне.

Если данные введены правильно, то программа выведет результат в нижнем текстовом поле.

\ 
\ssec{Инструкция программисту}
При разработке программы был определен тип TFunc=function(x:Double):Double - тип вспомогательных функций.
При разработке программы построение интерполяционного многочлена Ньютона с разделенными разностями были написаны следующие процедуры и функции:
%\suppressfloats[p]
\newcommand{\ftab}[1]{
 представлены в таблице \ref{#1}
}
\elist{
\item Процедура btnRunClick - основная процедура программы расчета сложной функции.

Параметры процедуры \ftab{btnRunClick:1}:
\tabl{Параметры основной процедуры программы}
{Sender&TObject&объект-возбудитель события \\ \hline
}{btnRunClick:1}{H}

Локальные переменные  процедуры \ftab{btnRunClick:2}:
\tabl{Локальные переменные основной процедуры программы}{
\tabln{x,y &Double& точка для расчета значения сложной функции}
\tabln{f &TFunc& \parbox{10cm}{функция, с помощью которой рассчитывается значение сложной функции}}
\tabln{d &Double& значение функции.}
}{btnRunClick:2}{H}
\item Функция Input вводит данные для расчета значения сложной функции.

Возвращает True, если операция прошла успешно, иначе возвращает False.

В функции обьявляются константы n=3 - количество вспомогательных функций, и
TestFuncs:Array[0..n-1] of TFunc = (f1,f2,f3) - массив вспомогательных функций.
function TfrmMain.Input(var x,y:Double;var f:TFunc):boolean;

Параметры функции \ftab{input:1}:
\tabl{Параметры функции   ввода данных}{
\tabln{x,y &Double& точка для расчета значения сложной функции}
\tabln{f &TFunc&\parbox{10cm}{ функция, с помощью которой рассчитывается значение сложной функции}}
}{input:1}{H}

Локальные переменные  функции \ftab{input:2}:
\tabl{Локальные переменные  функции  ввода данных}{
\tabln{errors &string& список информации о возникших ошибках I/O.}
}{input:2}{H}
\item Функция CalcD рассчитывает значение сложной функции.

Возвращает рассчитанное значение.

function CalcD(x,y:Double;f:TFunc):Double;

Параметры функции \ftab{calcd:1}:
\tabl{Параметры функции  расчета  значения сложной функции}{
\tabln{x,y &double& точка для расчета значения сложной функции}
\tabln{f &TFunc&\parbox{10cm}{ функция, с помощью которой рассчитывается значение сложной функции}}
}{calcd:1}{H}

\item Процедура Output выводит рассчитанное значение сложной функции на форму.

TfrmMain.Output(d:Double);

Параметры  процедуры \ftab{output:1}:
\tabl{Параметры процедуры вывода значения функции}{
\tabln{d &Double& значение сложной функции}
}{output:1}{H}
}
\clearpage
\ssec{Текст программы}
Ниже представлен текст программы на языке Delphi 7, реализующей расчет  значения сложной функции.

\prog{Delphi}{UnitMain.pas}

\ 
\ssec{Тестовый пример}
Ниже на рисунке \ref{scr:1} представлен пример работы программы при x=0, y=-1, и вспомогательной функции $f(x)=e^x$.
\pic{SCR1.png}{Пример работы программы  c правильными исходными данными}{scr:1}{H}
На рисунке \ref{scr:2} представлен пример работы программи с неправильными исходными данными.
\pic{SCR2.png}{Пример работы программы  c правильными исходными данными}{scr:2}{H}
\ 
\ssec{Вывод}
В этой лабораторной работе я изучил операторы ветвления и циклов Delphi и компонент TMaskEdit. Именно операторы ветвления и циклов позволяют программе изменять поведение при получении внешних данных, а компонент TMaskEdit помогает организовать их безопасное получение.
\end{document}
