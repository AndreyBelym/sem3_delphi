\input template.tex
\begin{document}
\selectlanguage{russian}
\maketitle 7 {ИНТЕРФЕЙС ПЕРЕНОСА DRAG-AND-DROP}
\setcounter{page}{2}
\normalfont
\ssec{Цель работы}
Цель работы заключается в том, чтобы изучить интерфейс переноса Drag\&Drop и написать программу с его использованием.  

\ 
\ssec{Задание на работу}
Реализовать конвертер единиц измерения. На форме расположить список доступных единиц измерения. Из него выбирать нужные единицы измерения путем перетаскивания их мышкой. Результат перевода вместе с исходными данными фиксировать в TMemo, в виде строки, например, 15 сантиметров = 150 миллиметров. 

\clearpage
\ssec{Теоретическая справка}
{\hbadness=10000
\loop{}\ifnum\thepage<5{\hrule\hfill\\}\repeat
\newcounter{lc}\setcounter{lc}{29}
\loop{}\ifnum\value{lc}>0{\addtocounter{lc}{-1}\hrule\hfill\\}\repeat
}
%\vfill
\ssec{Инструкция пользователю}
 Программа позволяет конвертировать величины различных единиц измерения между собой.

 Выберите класс  единиц измерения, после чего перетащите наименования единиц измерения в области, ограниченные рамкой. Введите значение в любое поле ввода и нажмите клавишу <Enter> на клавиатуре или кнопку "=" в окне программы.
\ 
\ssec{Инструкция программиста}
При создании программы конвертации единиц измерения были объявлены типы:
\elist{
\item TConvertUnits - запись, описывающая список типов(единиц измерения) величин.

В таблице \ref{TConvertUnits:2} представлены поля этого типа:
\tabl{Поля типа  "список единиц измерения"}{
\tabln{Category&ShortString&категория}
\tabln{Units&array of ShortString&наименования}
}{TConvertUnits:2}{H}
   
    
\item  TUnitsTable=array of TConvertUnits, описывает таблица типов величин.

\item  TConvertFunction=function (value:extended):extended; stdcall;

описывает тип функции перевода величин.

\item  TConvertLine - запись, которая описывает строку таблицы конвертации.
В таблице \ref{TConvertLine:2} приведены поля этого типа.
\tabl{Поля типа  " строка таблицы конвертации"}{
\tabln{SrcUnit,DstUnit&ShortString &исходный и требуемый типы}
\tabln{ConvertFunc&TConvertFunction& функция конвертации}
}{TConvertLine:2}{H}
    
    

\item  TConvertTable=array of TConvertLine, описывает таблицу конвертации.

 
\item  UnitsTableExportFunc=function:TUnitsTable; stdcall;

описывает тип функции экспорта таблицы типов из DLL.
  
\item  ConvertTableExportFunc=function:TConvertTable;  stdcall;

описывает тип функции экспорта таблицы конвертации из DLL
  
\item  GarbageCollectorFunc=procedure (var units:TUnitsTable;

var convert:TConvertTable); stdcall;

тип функции сборки мусора из DLL
}


Были объявлены следующие глобальные переменные, описание которых приводится в таблице \ref{GlobalVar:1}:


\tabl{Глобальные переменные модуля}{
\tabln{ ConvertTable&TConvertTable&таблица конвертации}
\tabln{  UnitsTable&TUnitsTable&таблица единиц измерения}
\tabln{  LoadedLibraries&array of Cardina&список подключенных DLL}
}{GlobalVar:1}{H}

Далее программа была разбита на следующие подпрограммы:

\elist{
\item Функция GetConvFunction получает из таблицы ConvTable для единиц измерения
SrcUnit(исходный тип) и DstUnit(целевой тип) 

функцию конвертации ConvFunc.

Если функция найдена, возвращает True, в противном случае возвращает False.

function GetConvFunction(ConvTable:array of TConvertLine;
          
SrcUnit,DstUnit:ShortString;
           
 var ConvFunc:TConvertFunction):Boolean;

Параметры  функции \ftab{GetConvFunction:1}:
\tabl{Параметры  функции получения функции конвертации}{
\tabln{ConvTable&array of TConvertLine & таблица конвертации}
\tabln{SrcUnit,DstUnit &ShortString & исходный тип и целевой тип конвертации}
\tabln{var ConvFunc &TConvertFunction  & функция конвертации}
}{GetConvFunction:1}{H}

Локальные переменные  функции \ftab{GetConvFunction:2}:
\tabl{Локальные переменные  функции получения функции конвертации}{
\tabln{i &LongInt  & счетчик для обработки массива}
}{GetConvFunction:2}{H}

\item Процедура ConnectDLL подключает библиотеку DllName и
импортирует таблицы единиц измерений и конвертации.

procedure ConnectDLL(DllName:PChar);

Параметры процедуры \ftab{ConnectDLL:1}:
\tabl{Параметры процедуры подключения DLL}{
\tabln{DllName &PChar  & путь к библиотеке}
}{ConnectDLL:1}{H}


Локальные переменные  процедуры \ftab{ConnectDLL:2}:
\tabl{Локальные переменные  процедуры подключения DLL}{
\tabln{Handle &Cardinal  & дескриптор библиотеки}
\tabln{NewUnits &TUnitsTable  & импортированная таблица типов}
\tabln{NewConvert &TConvertTable &импортированная таблица конвертации}
\tabln{UExport &UnitsTableExportFunc  & функция импорта таблицы типов}
\tabln{CExport &ConvertTableExportFunc & функция импорта таблицы конвертации}
\tabln{GCollect &GarbageCollectorFunc  & функция сборки мусора}
\tabln{i,j,k,m &LongInt  & счетчики для обработки массивов;}
\tabln{l1,l2 &LongInt  &\parbox{7cm}{ размеры существующей и \\импортированной таблиц конвертации}}
\tabln{l1u,l2u &LongInt  & \parbox{7cm}{размер существующего \\списка типов и импортированного}}
\tabln{appended,exist &boolean  & \parbox{7cm}{флаги наличия импорти-\\руемых данных в таблицах}}
}{ConnectDLL:2}{H}

\item Процедура lblFstUnitDragDrop -  обработчик "сбрасывания" в поле единиц измерения.

  procedure TfrmMain.lblFstUnitDragDrop(Sender, Source: TObject; X, Y: Integer);

Параметры процедуры \ftab{lblFstUnitDragDrop:1}:
\tabl{Параметры процедуры-обработчика события"сбрасывания" в поле единиц измерения}{
\tabln{Sender&TObject & объект-возбудитель события}
\tabln{Sender&TObject & объект-возбудитель события}
}{lblFstUnitDragDrop:1}{H}

\item  Процедура lblFstUnitDragOver  -  обработчик "попадания" в поле единиц измерения.

     procedure TfrmMain.lblFstUnitDragOver(Sender, Source: TObject; X, Y: Integer;
     

 State: TDragState; var Accept: Boolean);

Параметры процедуры \ftab{lblFstUnitDragOver:1}:
\tabl{Параметры процедуры-обработчика события "попадания" в поле единиц измерения}{
\tabln{Sender&TObject & объект-приемник и объект-источник DragNDrop}
\tabln{X, Y&Integer &координаты курсора}
\tabln{State&TDragState &состояние операции DragNDrop}
\tabln{Accept&Boolean & принять объект или нет}
}{lblFstUnitDragOver:1}{H}

\item  Процедура imgFstTrashDragOver  -  обработчик "сбрасывания" в мусорную корзину.

     procedure TfrmMain.imgFstTrashDragOver(Sender, Source: TObject;

 X, Y: Integer;
      
State: TDragState; var Accept: Boolean);

Параметры процедуры \ftab{imgFstTrashDragOver:1}:
\tabl{Параметры процедуры-обработчика события "сбрасывания" в мусорную корзину}{
\tabln{Sender&TObject & объект-приемник и объект-источник DragNDrop}
\tabln{X, Y&Integer &координаты курсора}
\tabln{State&TDragState &состояние операции DragNDrop}
\tabln{Accept&Boolean & принять объект или нет}
}{imgFstTrashDragOver:1}{H}

\item Процедура imgFstTrashDragDrop   -  обработчик "попадания" в мусорную корзину.

     procedure TfrmMain.imgFstTrashDragDrop(Sender, Source: TObject;

 X, Y: Integer);

Параметры процедуры \ftab{imgFstTrashDragDrop:1}:
\tabl{Параметры процедуры-обработчика события "попадания" в мусорную корзину}{
\tabln{Sender&TObject & объект-приемник и объект-источник DragNDrop}
\tabln{X, Y&Integer &координаты курсора}
}{imgFstTrashDragDrop:1}{H}

\item Процедура edtFstValKeyPress  -  обработчик нажатий клавиатуры в поле ввода величин.

     procedure TfrmMain.edtFstValKeyPress(Sender: TObject; var Key: Char);

Параметры процедуры \ftab{edtFstValKeyPress:1}:
\tabl{Параметры процедуры-обработчика события нажатий клавиатуры в поле ввода величин}{
\tabln{Sender&TObject & объект-возбудитель события}
\tabln{Key&Char & символ нажатой клавиши}
}{edtFstValKeyPress:1}{H}

\item   Процедура cbxCategorySelect -  обработчик выбора категорий типов.

     procedure TfrmMain.cbxCategorySelect(Sender: TObject);

Параметры процедуры \ftab{cbxCategorySelect:1}:
\tabl{Параметры процедуры-обработчика события нажатия  выбора категорий типов}{
\tabln{Sender&TObject & объект-возбудитель события}
}{cbxCategorySelect:1}{H}

 \item Процедура FormCreate  - обработчик создания формы.
  
   procedure TfrmMain.FormCreate(Sender: TObject);

Параметры процедуры \ftab{FormCreate:1}:
\tabl{Параметры процедуры-обработчика события создания формы}{
\tabln{Sender&TObject & объект-возбудитель события}
}{FormCreate:1}{H}

\item  Процедура FormDestroy  -  обработчик разрушения формы.

     procedure TfrmMain.FormDestroy(Sender: TObject);

Параметры процедуры \ftab{FormDestroy:1}:
\tabl{Параметры процедуры-обработчика события разрушения формы}{
\tabln{Sender&TObject & объект-возбудитель события}
}{FormDestroy:1}{H}

\item Процедура btnEquClick -   обработчик нажатия кнопки "равно".

    procedure TfrmMain.btnEquClick(Sender: TObject);

Параметры процедуры \ftab{btnEquClick:1}:
\tabl{Параметры процедуры-обработчика события нажатия кнопки "равно"}{
\tabln{Sender&TObject & объект-возбудитель события}
}{btnEquClick:1}{H}
}

Кроме того, подключаемые DLL должны экспортировать функции:
\elist{
\item Процедура CollectGarbage освобождает память из-под таблицы
единиц измерения units и таблицы конвертации convert.

procedure CollectGarbage(var units:TUnitsTable;

var convert:TConvertTable); stdcall;

\item Функция ExportUnitsTable возвращает экспортируемую библиотекой
таблицу единиц измерения.

function ExportUnitsTable:TUnitsTable; stdcall;

\item Функция ExportConvertTable возвращает экспортируемую библиотекой
таблицу единиц конвертации.

function ExportConvertTable:TConvertTable; stdcall;

}
\clearpage
\ssec{Текст программы}
Далее приведен текст программы конвертации единиц измерения, написанной на языке Delphi 7.
\prog{Delphi}{UnitMain.pas}

Ниже приведен текст библиотеки для конвертации единиц температуры.
\prog{Delphi}{thermal.dpr}

Ниже приведен текст библиотеки для конвертации единиц давления.
\prog{Delphi}{pressure.dpr}

Ниже приведен текст библиотеки для конвертации единиц энергии.
\prog{Delphi}{energy.dpr}

\clearpage

\ssec{Тестовый пример}
Ниже на рисунке \ref{SCR1} представлен пример работы программы конвертации единиц измерения.
\pic{SCR1.png}{Пример работы программы  конвертации единиц измерения}{SCR1}{H}
\ 
\ssec{Вывод}
В этой лабораторной работе я изучил способы реализации механизма Drag-and-Drop в программах на Delphi. Механизм  Drag-and-Drop позволяет создавать удобные и дружелюбные к пользователю приложения, т. к. позволяет сделать интерфейс приложения более интуитивным.
\end{document}
